\section{エディタの進め}
\subsection{今まで}
今まではエディタとしてgeditを使いgccのコンパイラを使い一つ一つ実行していきました.\\
ただ今週から私も自分のPCを使って講義を行っていきますし皆さんとは違う環境でプログラムを書きます.\\
そこで皆さんにオススメのエディタや統合開発環境をここに書いておきます.\\
\subsection{エディタと統合開発環境の違い}
簡単に言うとエディタはgeditのように文字を入力することしかできないもので統合開発環境はコンパイラとセットになった便利なものです.\\
私が行った講座ではエディタとコンパイラを別で扱っていましたが統合開発環境の方がやりやすい場合も多々あります.\\
なので次からはおすすめの統合開発環境について書いていきます.
\section{おすすめの統合開発環境}
\subsection{Windows}
\subsubsection{Visual Studio2019}
すごく有名な統合開発環境\\
補完機能はレベルがそこまで高くないらしいがMSが作った統合開発環境なのでとてもレベルが高い.\\
また様々な言語に対応しているためとても便利.\\
ただしscanf()関数が通らないのでscanf\_ s()という関数に変更する必要がある.\\
\subsubsection{InteliJ IDEA}
とても使いやすいらしい.\\
Javaの統合開発環境としてのイメージが強いがプラグインでC言語などに対応させられる.\\
\subsection{MacOS}
\subsubsection{Xcode}
Appleが公式で出している統合開発環境\\
ただし重い・補完機能がそこまでないなどの弱点も抱えている.\\
ただAppleの言語(SwiftやObjective-Cなど)は強い.\\
\section{おすすめのエディタ}
\subsection{Windows}
\subsubsection{SublimeText}
エディタの文字が綺麗だったりシンタックスハイライトがすごく高速だったりでとても使いやすいエディタ\\
長い間使ってた.
\subsection{MacOS}
\subsubsection{SublimeText}
Windowsと同じ
\subsection{vim}
CUI上で動くエディタ.拡張性がとても高くなおかつ高速.\\
vimrcを編集することで様々なカスタマイズができる.\\
私は普段このエディタを使っている.\\
