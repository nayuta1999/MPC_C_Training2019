\section{演習問題}
\subsection{問1}
\subsubsection{解答例}
\lstinputlisting{\codepath/papyrus/exercise1.c}
\subsubsection{解説}
\begin{verbatim}
正直特に解説することがない。
この問題が解けないということは、考え方ではなく関数への理解が足りていないと思われる。
関数は慣れればとても便利なものなので、是非マスターしよう!
ちなみに「//return (a < b)? b : a;」は三項演算子を用いた方法である。

関数について: https://9cguide.appspot.com/11-01.html
\end{verbatim}

\subsection{問2}
\subsubsection{解答例}
\lstinputlisting{\codepath/papyrus/exercise2.c}
\subsubsection{解説}
\begin{verbatim}
累乗を計算するというもの。
今回気を付けるべきはループする回数、
0乗(n = 0)のときの処理、
変数の扱える値を超えること(オーバーフロー)への注意の3点だろうか。

ループの判定がnではなくn-1となっている点に注意していただきたい。
0乗は1になるので、if文で分岐し、return文で返してしまおう。
return文が実行されると、
それ以降の文は実行されずに、その関数が閉じられることを理解しよう。
値の扱える範囲はいつも注意しなければならない。
今回の入力範囲では、最大100の10乗が考えられ、int型では扱えないことを気を付けよう。
最終的な値だけでなく、途中で利用する変数などの値にも常に気を配ろう。

Cの扱える値の範囲: https://webkaru.net/clang/limits-header/
\end{verbatim}

\subsection{問3}
\subsubsection{解答例}
\lstinputlisting{\codepath/papyrus/exercise3.c}
\subsubsection{解説}
\begin{verbatim}
文字コードの知識がなければまず解けない問題である。
「'a' - 'A'」はaとAの文字コードの値の差を表している。
Cで扱うASCIIコード表において、'a' - 'A' = 32である。
大文字より小文字のほうが値が大きいことさえ知っていれば、解くことができる。
printf("%d\n", 'a' - 'A')は32を出力。
printf("%c\n", 'a' - 32)はAを出力する(31ならB)。

文字の扱い: https://9cguide.appspot.com/14-01.html
\end{verbatim}

\subsection{最後に}
\begin{verbatim}
今回作ってもらった3つの関数は標準ライブラリ関数としてCに実装されている。
以下リファレンス
fmax: http://www.c-tipsref.com/reference/math/fmax.html
pow: http://www.c-tipsref.com/reference/math/pow.html
tolower: http://www.c-tipsref.com/reference/ctype/tolower.html
toupper: http://www.c-tipsref.com/reference/ctype/toupper.html

\end{verbatim}
