\section{writer:榊 演習問題}
\begin{verbatim}
この文章は見なくて良いです。演習問題に対する榊くんのぼやき的なもの。
今回の問題の2次元配列を使う問題は、実際には使用しないで出力することが可能である。
なぜ配列を使うんだ...?と思った人は、そういう操作をする練習だと思ってほしい。
配列を使用しないと解けない問題は、難しくなりやすく、演習の範囲を超えてしまうからである。
EX問題は大不評のため廃止しました。
\end{verbatim}
\subsection{問1}
\begin{verbatim}
問1
出題意図:配列とfor文はセットで覚える!

6つの要素を持つint型の配列にscanfを用いて値を入力する。配列の要素を全て3倍し、その値を出力しなさい。
入力例:
1 6 38 2 198 51
出力例:
3 18 114 6 594 153
\end{verbatim}


\subsection{問2}
\begin{verbatim}
問2
出題意図:配列を覚えるまでは出来なかった、要素の逆順出力が可能となる。

問1の結果を逆順に出力せよ。
入力例:
1 6 38 2 198 51
出力例:
153 594 6 114 18 3
\end{verbatim}

\subsection{問3}
\begin{verbatim}
2×2の配列を2つ用意し、それぞれにscanfを用いて値を入力する。この2つの配列を行列として考える。
(行列がわからなければ、先輩に聞くか調べるかしましょう。)
与えられた2つの行列の和を求め、出力せよ。
入力例:
3 5
1 2

4 6
8 2

出力例:
7 11
9 4
\end{verbatim}

\subsection{問4}
\begin{verbatim}
問4
10個の要素を持つint型の配列を宣言し、scanfにより、要素を10個入力する。
この配列の要素全ての積を求めよ。大きな値を入れるとオーバーフローするが、気にしないで良い。

入力例1:
2 3 4 1 6 2 5 3 4 6
出力例1:
103680

入力例2:
10 20 30 40 50 60 70 80 90 100
出力例2:
1704722432
int型ではオーバーフローして、不適切な値が出力されます。
*本来10の倍数が出力されないとおかしいです。適切な値を表示したければ、下記のキーワードを参考に調べること。
先輩や知っている人に聞いても良い。
自己学習用検索ワード:「C言語 long long int」「オーバーフロー」「C言語 int型の最大値」
\end{verbatim}

\subsection{問5}
\begin{verbatim}
九九表を2次元配列に格納し、奇数の段のみ出力せよ。九九の表の出力方法は、第3回演習の解答を見て良い。
  1  2  3  4  5  6  7  8  9
  3  6  9 12 15 18 21 24 27
  5 10 15 20 25 30 35 40 45
  7 14 21 28 35 42 49 56 63
  9 18 27 36 45 54 63 72 81
このような表が出力されれば正解である。
\end{verbatim}

\subsection{演習問題が終わって暇な方へ}
\begin{verbatim}
W×H行列Aとw×h行列Bの積を求めるプログラムを作成しなさい。
計算できない場合はその旨を出力し、計算できる場合は計算結果を出力せよ。
この問題は難しいので解答は載せないが、演習問題が終わってしまって暇な人は取り組んでみると勉強になるかもしれない。
解を知りたい人は榊までお問い合わせください。インターネットで検索すればコードは見つかると思いますが。
\end{verbatim}
