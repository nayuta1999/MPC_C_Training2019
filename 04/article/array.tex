\section{配列}
 今までは、変数を使うときは一つずつ宣言してきました。ですが、100個、1000個などと個数の多いデータを取り扱いたいとき
に変数をそのぶん宣言するのは無理があります。
そんなときに便利なのが配列です。配列を使うと複数のデータをまとめて取り扱うことができます。
\subsection{配列の宣言について}
ということで今紹介した配列の宣言方法についての説明です。宣言方法は以下のようになっています。
\begin{itembox}{配列の宣言方法}
\begin{verbatim}
型名 配列名 [要素数];
例)int arry[5];
\end{verbatim}
\end{itembox}
配列名というのは、変数の名前と同じと考えて大丈夫です。要素数というのは、つくられる変数の数のことです。
ここで要素数として宣言できるのは自然数のみです。
\subsection{配列の扱い方例}
 配列の扱い方として以下のプログラムを実行してみましょう。
\lstinputlisting{\codepath/array1.c}
\begin{itembox}{実行結果}
array[0]=5\\
array[1]=6\\
array[2]=7
\end{itembox}
上のプログラムを見てわかるように、配列の要素数は0から始まります。つまり、arra[3]と宣言するとarray[0]~array[2]までがつくられるということです。

\subsection{配列の初期化、代入について}
 配列も宣言と同時に初期化を行うことができます。\\
以下のプログラムを実行してみましょう。実行結果は前のと同じなので省略します。
\lstinputlisting{\codepath/array1-1.c}

ちなみに以下のように宣言と同時に代入を行う場合、配列の要素数を省略することができます。
また、for文を利用して、配列の中身を以下のように確認することができます。これも実行結果は同じになります。

\lstinputlisting{\codepath/array2.c}
以下のような代入はエラーとなります。注意するようにしてください。
\lstinputlisting{\codepath/error.c}

ちなみに以下のようなこともできますわ。
\lstinputlisting{\codepath/array3.c}
\begin{itembox}{実行結果}
5 numbers put on please\\
5 6 7 8 9\\
5 6 7 8 9
\end{itembox}
