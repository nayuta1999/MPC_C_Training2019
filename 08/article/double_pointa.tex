\section{ポインタのポインタ}
ポインタはアドレスを保存するための変数ですが、そのポインタもメモリ上のどこかに保存されているので、アドレスを持っているということになります。
つまり、ある変数のポインタを保存しているメモリのアドレスを保存しているポインタがポインタのポインタです。(ダブルポインタ)
\begin{itembox}{ダブルポインタの宣言}
型名 **ポインタ名
\end{itembox}

\subsection{2次元配列を関数の引数にする}
前回は1次元配列を関数の引数にするということをポインタを用いて行いましたが、今回はダブルポインタを用いて2次元配列を関数の引数ということにします。

\lstinputlisting{\codepath/2-1.c}

\begin{itembox}{出力結果}
\begin{verbatim}
3 4
8 9
\end{verbatim}
\end{itembox}

