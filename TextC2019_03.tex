%------------------------------------- ページサイズなどの書式設定
%¥documentclass[a4j,twocolumn, dvipdfmx]{jsarticle} % 二段組の構成にする
%¥documentclass[a4j,notitlepage]{jsarticle} % タイトルだけのページを作らない
\documentclass[a4j,titlepage,dvipdfmx]{jsarticle}   % タイトルだけのページを作る
%------------------------------------- パッケージ読み込み
\newcommand{\stypath}{./sty}
\newcommand{\codepath}{./code/03}
\usepackage[ipaex]{pxchfon}
%\usepackage{itembkbx}
\usepackage{\stypath/listings}
\usepackage{ascmac}
\usepackage{\stypath/jlisting}
\lstset{% 
showstringspaces=false,%空白文字削除
language={C},% %言語選択
basicstyle={\upshape},% %標準の書体
identifierstyle={\small},% %キーワードでない文字の書体
ndkeywordstyle={\small},% %キーワードその2の書体
stringstyle={\small\ttfamily},% %””で囲まれた文字などの書体
frame={tb},% %枠、デザインなど
breaklines=true,% %行が長くなった時の自動改行
columns=[l]{fullflexible},% %書体による列幅の違いを調整するか
numbers=left,% %行番号を表示するか
xrightmargin=0zw,% %余白の調整?
xleftmargin=0zw,% %余白の調整
numberstyle={\scriptsize},%行番号の書体
stepnumber=1,% %行番号をいくつ飛ばしで表示するか
numbersep=1zw,% %行番号と本文の間隔
morecomment=[l]{//}% 
} 
\title{C言語講座第二回}
\author{那由多(堀越亮我)}
\date{2019年5月16日}
\begin{document}
\maketitle
\section{前回の復習}
\begin{enumerate}
\item if,if-else if-else if-else
\item switch-case
\end{enumerate}
\section{値の代入操作とインクリメント・デクリメント}
\subsection{値の代入操作}
第一回の時にC言語における=は左の変数に右の数式の計算結果を代入するという話をしました.\\
それの復習を行います.
\begin{lstlisting}
number=334;
\end{lstlisting}
この代入操作には何も違和感がありませんが次の場合はどうでしょう.
\begin{lstlisting}
number=number+1;
\end{lstlisting}
この代入操作はおかしく見えますが正しい操作です.これはnumberに+1した数字をnumberに代入するという操作です.
この操作は少し省略して書くことができます.

\begin{lstlisting}
number+=1
\end{lstlisting}
この表記は先ほどと同じ操作です.
具体的には
\begin{itembox}
"変数" "計算操作"="変数を操作したい値"
\begin{itembox}
という風になります.
少し見辛いですがよく使う表現ですのでよく確認しておきましょう

\subsection{インクリメント・デクリメント}
%ここからはまだ
\subsection{前置と後置}
\section{繰り返し操作}
\section{制御}
\section{演習問題}
\end{document}