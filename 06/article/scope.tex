\section{関数のスコープについて}
\subsection{スコープ}
 変数は宣言した場所によって湯広範囲がことなります。この有効範囲のことをスコープといいます。
\lstinputlisting{\codepath/scope1.c}
ちなみにこのプログラムはエラーになります。。
これはmain関数内の変数numとfunc関数内では名前が同じnumという変数でも別物とみられているわけです。
裏を返せばこのようにiがmain関数とfunc関数に現れても問題ないということです。
\lstinputlisting{\codepath/scope2.c}

\subsection{グローバル変数}
 関数の外で変数を宣言することによってすべての関数から使用することができるグローバル関数というものもあります。
(逆に関数内で宣言する変数はローカル変数と呼ばれます)
\lstinputlisting{\codepath/glo.c}
一見グローバル変数は便利に見えますが、何をするための変数なのかわからなくなったり、すべての変数からアクセスできるという点から
バグや予期せぬ処理の温床になるので"どうしても使いたい場面以外では使わない"というのが一般論です。(ある授業ではグローバル変数が沸きます)