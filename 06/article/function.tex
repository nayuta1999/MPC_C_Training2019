前回が復習会だったので今回は復習を省略します。
\section{関数について}
 関数とは値を入れるとそれに対応した値を返すもの、複数の操作を行ったりするようなものです。
ここでは自分が使いやすい、便利と思うような関数を定義する方法をやっていきます。

\subsection{いろいろな関数}
早速ですが例文です。

\lstinputlisting{\codepath/func1.c}

関数に与える値、()内で関数に与えるデータを引数(ひきすう)といい、居れた値やデータに対する値を返り値といいます。
また、プログラムの最初の3行を見ると分かりますが、includeされているものが増えていることが分かります。
これは使える関数を増やしている、拡張しているということです。
この.hがついているものをヘッダーファイルといいます。

\section{関数の使い方}
 関数は以下のように使います。また、返り値を持たない関数の型はvoidを使います。
\begin{itembox}{宣言のやり方}
\begin{verbatim}
返り値の型 変数名 (引数1,引数2,…){
    関数の処理内容
}
\end{verbatim}
\end{itembox}
それではここからはある程度パターンを分けてみていきましょう。

\subsection
 下のようなプログラムがあったとします。

\lstinputlisting{\codepath/func2.c}
見てわかるように、このプログラムは同じことを何度も書いているためとても非効率です。
以下のように関数化してみましょう。
\lstinputlisting{\codepath/func3.c}
今回の例では関数かをすることによって大きな効果を得ることができていませんが、コードの長さが長くなるほど効果的になることが多いです。
また、関数としていない場合は同じ処理をかいたときにそれぞれの処理を操作しないと同じ動きをするように修正できませんが、
関数化した後は関数の中をいじるだけで簡単に修正できることが多いです。
\subsection{値を返す変数}
次は値を返す変数についてですがこれも先ほどのものとほぼ同様ですのでサンプルのみ。
\lstinputlisting{\codepath/func4.c}
\subsection{引数を取る関数・値を返す関数}
 次のようなプログラムがあったとします。
\lstinputlisting{\codepath/func5.c}
この計算部分を関数化してスマートにしてみましょう。

\lstinputlisting{\codepath/func6.c}

このプログラムは1~5までの合計値を計算するプログラムです。まず、関数を宣言している行(3行目)を見ると()内にint numとあります。
このような関数宣言に書かれた引数の型と名前のことを仮引数といいます。
続けて関数内をいていくと、10行目にreturn sumとあります。
これがいわゆる関数の返り値です。
関数の呼び出しの方は16行目のprintf内で行っています。
この呼び出しの時の()内の値(今回は変数N)、これを実引数と呼びます。

\subsection{複数の引数を持つ関数}
 ここでは2つの引数を持つ関数をつくってみます。

\lstinputlisting{\codepath/func7.c}

引数を2つ持つ関数は、宣言時に仮引数をint a,int bと2つ用意して、実引数もそれに対応させて2つ用意させれば良いのです。
ちなみにこれは引数が3つ、4つと増えても同じです。

\subsection{プロトタイプ宣言}
 ここまでやった自作の関数を呼び出したいとき、"呼び出す関数は使用する前に記述しておく"いうルールがあります。
このプロトタイプ宣言を使うことによって、中身の処理は記述せずに、こんな関数をこれから使うゾということをあらかじめ宣言したうえで、
下のほうに関数の処理を書いておくということができます。
以下のプログラムの例がそうです。
\lstinputlisting{\codepath/func8.c}








