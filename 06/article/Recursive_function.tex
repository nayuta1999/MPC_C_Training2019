\section{再帰関数について}
\subsection{再帰関数}

 関数は処理中に自分自身を呼び出すことができます。
そしてそれを再帰呼び出しといいます。以下のプログラムでは例として階乗の計算を行っています。
\lstinputlisting{\codepath/func9.c}
factorial関数内の処理を見ると自分自身を呼び出しています。再帰関数はループ文と同様に終了処理を与えないと無限ループになったりするので注意してください。
今回はfactorial(10)より
10!=10*(9!)   9! =9*(8!)    8!=8(7!)…のようになっています。

今回はxが0となると1を返すようになっています。0!=1ということです。
ちなみに勘の鋭い方は"これループでいいじゃん"って思いますよね。その通りです。余裕があれば書いてみてください。
