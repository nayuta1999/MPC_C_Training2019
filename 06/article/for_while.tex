\section{ループ文}
\subsection{for文}
 コンピュータにい待った回数同じ操作を行ってほしいというときにfor文を使います.
まずは下のプログラムを実行してみましょう.


\lstinputlisting{\codepath/for.c}

\begin{itembox}{実行結果}
0 1 2 3 4 5 6 7 8 9

\end{itembox}
このプログラムを見てわかるように,i++のおかげでiが1回ループするごとに1増加していることがわかります.
また,for文の基本的な内容はこのように泣ており,条件式を満たすまでループを行うという仕組みです.
変化式は値の変化を表すi++やi--が入ります.

\begin{itembox}{for文の書き方}
for(初期値;条件式;値の変化式)\{ \\
~~~~実行したい処理など;\\
\}
\end{itembox}
\subsection{while文}

\subsection{do-while文}
