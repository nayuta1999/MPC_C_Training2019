\section{演習問題(さんたろー作)}
\subsection{問1}
\subsubsection{解答例}
\lstinputlisting{\codepath/papyrus/exercise1.c}
\subsubsection{解説}
\begin{verbatim}
参考文献参照。

\end{verbatim}

\subsection{問2}
\subsubsection{解答例}
\lstinputlisting{\codepath/papyrus/exercise2.c}
\subsubsection{解説}
\begin{verbatim}
配列名だけ(a)は配列の先頭アドレスを表していることを理解しよう。
参考文献参照。

\end{verbatim}

\subsection{問3}
\subsubsection{解答例}
\lstinputlisting{\codepath/papyrus/exercise3.c}
\subsubsection{解説}
\begin{verbatim}
値渡しと参照渡しの違いを理解しよう。
引数で渡されたアドレスにあるint型の変数の値そのものが、関数によって書き換えられているのである。
参考文献参照。
\end{verbatim}

\subsection{問4}
\subsubsection{解答例}
\lstinputlisting{\codepath/papyrus/exercise4.c}
\subsubsection{解説}
\begin{verbatim}
パッと見た感じでは、配列はどこで書いても元の値が書き換わるように見えるかもしれない。
しかし、配列の先頭アドレスを参照渡しで渡しているためだということに気づいてほしい。
理解できない人は参考文献参照の上、配列の要素をint型として関数に渡し、関数内でその値を書き換えてみよう。
元の値が変わっていないことがわかるはずだ。
\end{verbatim}

\subsection{問5}
\subsubsection{解答例}
\lstinputlisting{\codepath/papyrus/exercise5.c}
\subsubsection{解説}
\begin{verbatim}
4ができれば、基礎を理解している人ならそんなに難しくないはずだ(ひな形あるし)。
C99から変数も配列の要素数として指定できるようになった。

\end{verbatim}
