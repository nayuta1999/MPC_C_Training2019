\section{榊作成 演習問題}
\begin{verbatim}
復習回です。今のうちにわからないところを洗い出しましょう。
わからなければ先輩に聞くか、調べるか、周りのわかっている人に聞きましょう。
大体難易度順ですが、わからなければ飛ばして次の問題に進みましょう。
EX問題は大不評のため、やめました。

要素数Nは1から100までの間の数で、キーボードからプログラムの初めに入力するものとする。
環境によってはint a[n];は出来ないので、配列の宣言はint a[100];などとすること。
(解答のコードは出来ない環境でコーディングしたので、int a[100]などとしている)
\end{verbatim}

\subsection{問1}
\begin{verbatim}
「This is where I'm going to buid my house.」と出力せよ。
どうでもよいが、日本語訳は「ここに我が家を建てよう。」です。
\end{verbatim}

\subsection{問2}
\begin{verbatim}
5*(3-1)+6.0/4を求め、出力せよ。答えは11.5である。
\end{verbatim}

\subsection{問3}
\begin{verbatim}
小数を2つ入力し、その和と積を出力せよ。

入力例:
1.414
1.414
出力例:
addition is 2.828000.
multiplication is 1.999396.
\end{verbatim}

\subsection{問4}
\begin{verbatim}
上底、下底、高さを入力し、台形の面積を求めるプログラムを作成せよ。
上底、下底、高さの順に入力する。
答えは少数になり得る点に注意せよ。台形の面積の公式を忘れた人は調べよ。
なお、この問題は第1回の演習と同じである。

入力例:
3
4
7

出力例:
24.500000
\end{verbatim}

\subsection{問5}
\begin{verbatim}
1から12までの数字を1つ入力し,
2以上4以下ならば「spring」と,
5以上7以下ならば「summer」と,
8以上10以下ならば「autumn」と,
それ以外は「winter」と,
出力しなさい。

余力があれば、1から12以外の数字が与えられた場合に「error!」と出力するように改造してみましょう。
\end{verbatim}

\subsection{問6}
\begin{verbatim}
「Make America Great Again!」と29回出力せよ。
これはトランプ大統領が定期的に呟くツイートである。
直訳すると「アメリカを再び偉大に!」という意味である。
\end{verbatim}

\subsection{問7}
\begin{verbatim}
要素を7つ持つint型の1次元配列を宣言し、適当な値を入力し、2倍にして出力せよ。
見やすくするため、各要素毎に改行せよ。

入力例:
2348 3527 549 382 2919 4395 382

出力例:
4696
7054
1098
764
5838
8790
764
\end{verbatim}

\subsection{問8}
\begin{verbatim}
N個の要素を持つint型配列を作り、その中に適当な値を入力する。入力した値の中での最小値と最大値を求め、出力しなさい。
入力例1:
5
14526 32  123 4 63235
出力例1:
min=4,max=63235.

入力例2:
1
100
出力例2:
min=100,max=100.
\end{verbatim}

\subsection{問9}
\begin{verbatim}
N×Nの単位行列を出力せよ。
大雑把な説明:対角成分が1で、その他が0である行列

入力例:
10

出力例:
1 0 0 0 0 0 0 0 0 0
0 1 0 0 0 0 0 0 0 0
0 0 1 0 0 0 0 0 0 0
0 0 0 1 0 0 0 0 0 0
0 0 0 0 1 0 0 0 0 0
0 0 0 0 0 1 0 0 0 0
0 0 0 0 0 0 1 0 0 0
0 0 0 0 0 0 0 1 0 0
0 0 0 0 0 0 0 0 1 0
0 0 0 0 0 0 0 0 0 1
\end{verbatim}

\subsection{問10}
\begin{verbatim}
N×N行列(2次元配列)に値を入力し、その行列の対角成分の和を求めなさい。

入力例:
2
183 329
382 471

出力例:
sum=654
\end{verbatim}

\subsection{問11}
\begin{verbatim}
任意のN個の整数を、要素数Nの配列に格納し、降順に並べ替えなさい。
要素の挿入は、rand()を用いる。以下のひな型ではNの値は固定してある。
バブルソート・選択ソート・挿入ソート・クイックソート・マージソートetc...
様々な手法があるので、調べて実装してみよう。

入力例1:
6
238 127 1349 427 1 4728
出力例1:
4728 1349 427 238 127 1

入力例2:
7
2 4 6 8 10 12 14
出力例2:
14 12 10 8 6 4 2
\end{verbatim}

\subsection{問12}
\begin{verbatim}
榊くんのプレイしているネトゲで「AtCoder」というものがあります。
ここ最近は参加人数のインフレが凄まじく、参加者は1か月当たり1.3倍になっています。(これは正確なデータではありません。)
ところで、1か月毎に参加人数が1.3倍になるとすると、世界人口70億人すべてがAtCoderをプレイする日が来るはずです。
ここで、70億はint型ではオーバーフローすることに注意し、何か月後にAtCoderの参加者が世界人口を超えるかを求めるプログラムを作成してください。double型で解を導けることは確認済みです。
現在の参加人数を5500人とし、1.3倍をした時の小数点以下も計算に含めること。
上記の条件において、解は一意に定まり、54と出力されれば正解である。
\end{verbatim}

\subsection{問13}
\begin{verbatim}
榊くんはゾロ目が好きです。
そこで、三桁の整数X(100<X<999)を貰った時、Xをゾロ目になるまでインクリメントします。
榊くんがゾロ目になるまでインクリメントした回数を出力しなさい。

入力例1:
111
出力例1:
0

入力例2:
561
出力例2:
105

入力例3:
311
出力例3:
22
\end{verbatim}

\subsection{問14}
\begin{verbatim}
1以上の整数が10個与えられます。以下の操作が何回行われるかを求めてください。
・全ての整数が偶数である限り、全ての整数を2で割る操作をする。

入力例1:
2 2 2 2 2 2 2 2 2 1
出力例1:
0

最初から2で割れません。操作は行われないので0です。

入力例2:
12 28 8 16 32 64 128 256 512 1024

出力例2:
2

まず、すべての整数は偶数です。よって、全ての整数を2で割ります。
6 14 4 8 16 32 64 128 256 512
まだ全て偶数なので全ての整数を2で割ります。
3 7 2 4 8 16 32 64 128 256
奇数が存在するのでこれ以上操作はできません。
よって、2を出力します。
\end{verbatim}

\subsection{問15}
\begin{verbatim}
この問題は、問11を解き、要素を昇順か降順に並べ替えるアルゴリズムを勉強した後に解くことをお勧めします。
榊くんはグループ分けを指揮することになりました。
3N人の各々の戦闘力が与えられます。
3人を1つのグループとします。つまり、グループはN個できます。
以下にグループ全体としての戦闘力の定義を記載します。
・その3人の戦闘力の中央値
例えば5 28 21というグループを作った時、そのグループの戦闘力は21です。
同様に、5,5,5のグループの戦闘力は5で、8,7,7のグループの戦闘力は7です。

このように定義したとき、N個のグループ全体において考えられる、グループの戦闘力の最大値と、最小値を求めてください。

入力例1:
10
\end{verbatim}
1248 1238 32 812 1238 342 428 4529 5734 19 47238 4271 327 482713 4273 4299 173711348 327 127 429488 127 128 38461 1738 61748 2618386 34869 362818 46372 18374638
\begin{verbatim}
出力例1:
MAX=18374638,MIN=32

入力例2:
1
123 3839 1

出力例2:
MAX=123,MIN=123
\end{verbatim}
