%------------------------------------- ページサイズなどの書式設定
%¥documentclass[a4j,twocolumn, dvipdfmx]{jsarticle} % 二段組の構成にする
%¥documentclass[a4j,notitlepage]{jsarticle} % タイトルだけのページを作らない
\documentclass[a4j,titlepage,dvipdfmx]{jsarticle}   % タイトルだけのページを作る
%-------------------------------------コマンド定義
%styファイルのパスの簡略化
\newcommand{\stypath}{./sty}
%コードファイルの簡略化(./code/04のように毎回変更する)
\newcommand{\codepath}{./code}
%記事ファイルの簡略化(codepathと同様)
\newcommand{\articlepath}{./article}
%------------------------------------- パッケージ読み込み
\usepackage[ipaex]{pxchfon}
%\usepackage{itembkbx}
\usepackage{\stypath/listings}
\usepackage{ascmac}
\usepackage{\stypath/jlisting}
\lstset{% 
showstringspaces=false,%空白文字削除
language={C},% %言語選択
basicstyle={\upshape},% %標準の書体
identifierstyle={\small},% %キーワードでない文字の書体
ndkeywordstyle={\small},% %キーワードその2の書体
stringstyle={\small\ttfamily},% %””で囲まれた文字などの書体
frame={tb},% %枠、デザインなど
breaklines=true,% %行が長くなった時の自動改行
columns=[l]{fullflexible},% %書体による列幅の違いを調整するか
numbers=left,% %行番号を表示するか
xrightmargin=0zw,% %余白の調整?
xleftmargin=0zw,% %余白の調整
numberstyle={\scriptsize},%行番号の書体
stepnumber=1,% %行番号をいくつ飛ばしで表示するか
numbersep=1zw,% %行番号と本文の間隔
morecomment=[l]{//}% 
} 

\title{C言語講座第5回解答}%何回か書き直す
\author{MPC部員}
\date{2019年6月6日}%日付も書き直す
\begin{document}
\maketitle
\section{演習問題(さんたろー作)}
\subsection{問1}
\subsubsection{解答例}
\lstinputlisting{\codepath/papyrus/exercise1.c}
\subsubsection{解説}
\begin{verbatim}
参考文献参照。

\end{verbatim}

\subsection{問2}
\subsubsection{解答例}
\lstinputlisting{\codepath/papyrus/exercise2.c}
\subsubsection{解説}
\begin{verbatim}
配列名だけ(a)は配列の先頭アドレスを表していることを理解しよう。
参考文献参照。

\end{verbatim}

\subsection{問3}
\subsubsection{解答例}
\lstinputlisting{\codepath/papyrus/exercise3.c}
\subsubsection{解説}
\begin{verbatim}
値渡しと参照渡しの違いを理解しよう。
引数で渡されたアドレスにあるint型の変数の値そのものが、関数によって書き換えられているのである。
参考文献参照。
\end{verbatim}

\subsection{問4}
\subsubsection{解答例}
\lstinputlisting{\codepath/papyrus/exercise4.c}
\subsubsection{解説}
\begin{verbatim}
パッと見た感じでは、配列はどこで書いても元の値が書き換わるように見えるかもしれない。
しかし、配列の先頭アドレスを参照渡しで渡しているためだということに気づいてほしい。
理解できない人は参考文献参照の上、配列の要素をint型として関数に渡し、関数内でその値を書き換えてみよう。
元の値が変わっていないことがわかるはずだ。
\end{verbatim}

\subsection{問5}
\subsubsection{解答例}
\lstinputlisting{\codepath/papyrus/exercise5.c}
\subsubsection{解説}
\begin{verbatim}
4ができれば、基礎を理解している人ならそんなに難しくないはずだ(ひな形あるし)。
C99から変数も配列の要素数として指定できるようになった。

\end{verbatim}

\section{演習問題}
\subsection{問1}
\subsubsection{問題}
\lstinputlisting{\codepath/sakaki/exercise1.c}
\subsubsection{解説}

\end{document}
