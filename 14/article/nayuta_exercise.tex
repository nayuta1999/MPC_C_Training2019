\section{演習問題(なゆた作成)}
\begin{enumerate}
\item 二つの数字が入力されます.swap関数を作成して実行した結果を表示しなさい.
\begin{itembox}{入力例}
\begin{verbatim}
2,4
\end{verbatim}
\end{itembox}

\begin{itembox}{実行例}
4,2
\end{itembox}
\item あなたはRPGのゲームを作ります.キャラクターのステータスは別のファイルとして用意します.構造体を使って適当なファイルから読み込んだ人間のステータスをまとめ,表示してください.また表示形式は実行例に従ってください.
\begin{itembox}{sample.txt}
\begin{verbatim}
H:20
A:50
D:20
\end{verbatim}
\end{itembox}

\begin{itembox}{出力結果}
\begin{verbatim}
HP:20
ATK:50
DEF:20
\end{verbatim}
\end{itembox}

\item \textbf{この問題は最後に解いてください}\\次のプログラムをしらべながら理解し,実行せよ.
\lstinputlisting{\codepath/cURL.c}
\begin{itembox}{実行結果}
\begin{verbatim}
{"message": "Internal server error"}
\end{verbatim}
\end{itembox}
\end{enumerate}